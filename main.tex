\documentclass{article}
\usepackage[utf8]{inputenc}
\usepackage[spanish]{babel}
\usepackage{geometry}
\usepackage{setspace}
\usepackage{graphicx}
\usepackage{xcolor}
\usepackage{tikz}
\usepackage{hyperref}
\usepackage{amsmath}
\usepackage{amssymb}
\usepackage{colortbl}
\usepackage{enumitem}

\geometry{a4paper, margin=2.5cm}
\definecolor{cenfotecblue}{RGB}{0,51,102}
\definecolor{cenfotecgray}{RGB}{128,128,128}

% Configuración de hyperref para enlaces
\hypersetup{
    colorlinks=true,
    linkcolor=cenfotecblue,
    urlcolor=cenfotecblue,
    citecolor=cenfotecblue,
    filecolor=cenfotecblue,
    pdfstartview=FitH
}

\title{\huge \textbf{Práctica 2}}
\author{\large Gabriel José Guzmán Leiva \\ \large Javier Pérez Arroyo}
\date{}

\begin{document}

\begin{titlepage}
\centering

% Logo en la parte superior con marco elegante
\vspace*{0.5cm}
\begin{tikzpicture}
\node[draw=cenfotecblue, line width=2pt, rounded corners=10pt, inner sep=15pt] {
  \includegraphics[width=8cm]{logo-cenfotec.png}
};
\end{tikzpicture}

\vspace{1.5cm}

% Línea divisoria superior
\textcolor{cenfotecblue}{\rule{0.9\textwidth}{3pt}}

\vspace{1cm}

% Título principal con sombra
{\Huge \textbf{\textcolor{cenfotecblue}{PRÁCTICA 2}}}

\vspace{0.8cm}

% Línea divisoria intermedia
\textcolor{cenfotecgray}{\rule{0.7\textwidth}{1pt}}

\vspace{1.5cm}

% Información del curso en caja
\colorbox{cenfotecblue!10}{\parbox{0.8\textwidth}{
\centering
\vspace{0.3cm}
{\Large \textbf{\textcolor{cenfotecblue}{FUN-06 Álgebra Lineal (MAT-04)}}}
\vspace{0.3cm}
}}

\vspace{2cm}

% Sección de autores con formato mejorado
{\large \textbf{\textcolor{cenfotecblue}{Elaborado por:}}}

\vspace{0.8cm}

\begin{tabular}{c}
{\large \textbf{Gabriel José Guzmán Leiva}} \\[0.5cm]
{\large \textbf{Javier Pérez Arroyo}}
\end{tabular}

\vspace{2cm}

% Información del profesor
{\large \textbf{\textcolor{cenfotecblue}{Profesor:}}}
\vspace{0.5cm}

{\large \textcolor{cenfotecgray}{Dorin Morales Monge}}

\vfill

% Pie de página con universidad y fecha
\textcolor{cenfotecblue}{\rule{0.6\textwidth}{2pt}}

\vspace{0.5cm}

{\Large \textbf{\textcolor{cenfotecblue}{Universidad Cenfotec}}}

\vspace{0.3cm}

\textcolor{cenfotecgray}{\rule{0.4\textwidth}{1pt}}

\vspace{0.3cm}

{\large \textbf{Octubre, 2025}}

\vspace{0.5cm}

\end{titlepage}

\newpage

% Índice con estilo personalizado
\begin{center}
{\Large \textbf{\textcolor{cenfotecblue}{ÍNDICE DE CONTENIDOS}}}
\end{center}

\vspace{1cm}

\begin{center}
\begin{tabular}{|c|p{8cm}|c|}
\hline
\rowcolor{cenfotecblue!20}
\textbf{Ejercicio} & \textbf{Descripción} & \textbf{Página} \\
\hline
\hyperlink{ejercicio1}{\textbf{Ejercicio 1}} & Análisis Vectorial de un Triángulo & \pageref{ejercicio1} \\
\hline
\hyperlink{ejercicio2}{\textbf{Ejercicio 2}} & [Título del Ejercicio 2] & \pageref{ejercicio2} \\
\hline
\hyperlink{ejercicio3}{\textbf{Ejercicio 3}} & [Título del Ejercicio 3] & \pageref{ejercicio3} \\
\hline
\hyperlink{ejercicio4}{\textbf{Ejercicio 4}} & Vectores en un videojuego 3D & \pageref{ejercicio4} \\
\hline
\end{tabular}
\end{center}

\vspace{1cm}

\begin{center}
\textcolor{cenfotecgray}{\textit{Haga clic en el número de ejercicio para navegar directamente}}
\end{center}

\newpage

% Ejercicio 1
\hypertarget{ejercicio1}{}
\label{ejercicio1}
\section*{Ejercicio 1: Análisis Vectorial de un Triángulo}

% Contenido del Ejercicio 1 sin preámbulo

\begin{center}
\colorbox{gray!10}{\parbox{0.95\textwidth}{
\vspace{0.3cm}
\centering
\textbf{\large Enunciado}
\vspace{0.2cm}

\begin{minipage}{0.9\textwidth}
Un triángulo está definido por los puntos $A(1,2,0)$, $B(4,6,0)$ y $C(3,2,5)$. A partir de esta información, realice lo siguiente:
\end{minipage}
\vspace{0.3cm}
}}
\end{center}

\begin{enumerate}[label=\textbf{\arabic*.}, itemsep=1em, leftmargin=2em]
    \item \textcolor{blue}{\textbf{Represente los vectores $\overrightarrow{AB}$, $\overrightarrow{AC}$, y $\overrightarrow{BC}$.}}

    \begin{center}
    \colorbox{blue!5}{\begin{minipage}{0.9\textwidth}
    \vspace{0.3cm}
    \textbf{Desarrollo para verificar los resultados:}
    \begin{gather*}
        \overrightarrow{AB} = (4-1,  6-2, 0-0) \\
        \overrightarrow{AB} = \boxed{(3, 4, 0)} \\[0.5em]
        \overrightarrow{AC} = (3-1,  2-2,  5-0) \\
        \overrightarrow{AC} = \boxed{(2, 0, 5)} \\[0.5em]
        \overrightarrow{BC} = (3-4,2-6,5-0) \\
        \overrightarrow{BC} = \boxed{(-1, -4, 5)}
    \end{gather*}
    \vspace{0.2cm}
    \end{minipage}}
    \end{center}
    
    \item \textcolor{blue}{\textbf{Calcule la longitud (norma) de cada vector.}}
    
    \begin{center}
    \colorbox{green!5}{\begin{minipage}{0.9\textwidth}
    \vspace{0.3cm}
    \textbf{Desarrollo de las normas:}
    \begin{align*}
        \|\vec{AB}\| &= \sqrt{3^2 + 4^2 + 0^2} = \sqrt{25} = \boxed{5} \\[0.8em]
        \|\vec{AC}\| &= \sqrt{2^2 + 0^2 + 5^2} = \sqrt{29} \approx \boxed{5.385} \\[0.8em]
        \|\vec{BC}\| &= \sqrt{(-1)^2 + (-4)^2 + 5^2} = \sqrt{42} \approx \boxed{6.480}
    \end{align*}
    \vspace{0.2cm}
    \end{minipage}}
    \end{center}
    
    \item \textcolor{blue}{\textbf{Determine algebraicamente el ángulo entre los vectores $\overrightarrow{AB}$ y $\overrightarrow{AC}$.}}
    
    \begin{center}
    \colorbox{red!5}{\begin{minipage}{0.9\textwidth}
    \vspace{0.3cm}
    \textbf{Desarrollo del producto punto:}
    \begin{align*}
        \vec{AB}\cdot\vec{AC} &= (3)(2) + (4)(0) + (0)(5) = 6 \\
        \cos \theta &= \frac{\vec{AB}\cdot\vec{AC}}{\|\vec{AB}\|\,\|\vec{AC}\|} = \frac{6}{5\sqrt{29}} \approx 0.223 \\
        \theta &= \cos^{-1}(0.223) \approx \boxed{77.1^\circ}
    \end{align*}
    \vspace{0.2cm}
    \end{minipage}}
    \end{center}

    \item \textcolor{blue}{\textbf{Interprete el resultado: ¿qué tipo de triángulo es?}}
    
    \begin{center}
    \colorbox{yellow!5}{\begin{minipage}{0.9\textwidth}
    \vspace{0.3cm}
    Como $77.1^\circ < 90^\circ$, el triángulo es \textbf{AGUDO}.
    \vspace{0.3cm}
    \end{minipage}}
    \end{center}
    
    \item \textcolor{blue}{\textbf{Represente el triángulo en Geogebra.}}
    
    \begin{center}
    \colorbox{gray!5}{\begin{minipage}{0.9\textwidth}
    \vspace{0.3cm}
    \begin{center}
    \textbf{Enlace:} \url{https://www.geogebra.org/3d/tmjswe7b}
    \end{center}
    \vspace{0.3cm}
    \end{minipage}}
    \end{center}
\end{enumerate}

\newpage

% Ejercicio 2 (placeholder)
\hypertarget{ejercicio2}{}
\label{ejercicio2}
\section*{Ejercicio 2}
\textit{[Contenido del Ejercicio 2 - Por desarrollar]}

\newpage

% Ejercicio 3 (placeholder)
\hypertarget{ejercicio3}{}
\label{ejercicio3}
\section*{Ejercicio 3}
\textit{[Contenido del Ejercicio 3 - Por desarrollar]}

\newpage

% Ejercicio 4
\hypertarget{ejercicio4}{}
\label{ejercicio4}
\section*{Ejercicio 4: Vectores en un videojuego 3D}

% Contenido del Ejercicio 4 sin preámbulo

\begin{center}
\colorbox{gray!10}{\parbox{0.95\textwidth}{
\vspace{0.3cm}
\centering
\textbf{\large Enunciado}
\vspace{0.2cm}

\begin{minipage}{0.9\textwidth}
En un videojuego 3D, un personaje se mueve en la dirección del vector
\[ \vec v_1 = (4,\,1,\,0) \]
y el viento aplica una fuerza representada por el vector
\[ \vec v_2 = (-1,\,2,\,0). \]
\end{minipage}
\vspace{0.3cm}
}}
\end{center}

\vspace{0.5em}
\noindent Realice lo siguiente:

\begin{enumerate}[label=\textbf{\arabic*.}, itemsep=0.8em, leftmargin=2em]
  \item Calcule el vector resultante del movimiento real del personaje $\big(\vec v_r = \vec v_1 + \vec v_2\big)$.
  \begin{center}
  \colorbox{blue!5}{\begin{minipage}{0.9\textwidth}
  \vspace{0.2cm}
  \textbf{Desarrollo:}
  \vspace{0.2cm}
  \begin{align*}
  \vec v_r &= \vec v_1 + \vec v_2 \\
       &= (4,1,0) + (-1,2,0) \\
       &= (3,3,0)
  \end{align*}
  \vspace{0.2cm}
  \end{minipage}}
  \end{center}
  
  \item Determine la velocidad total (norma de $\vec v_r$).
  \begin{center}
  \colorbox{green!5}{\begin{minipage}{0.9\textwidth}
  \vspace{0.2cm}
  \textbf{Desarrollo:}
  \vspace{0.2cm}
  \begin{align*}
  \|\vec v_r\| &= \sqrt{3^2 + 3^2 + 0^2} \\
         &= \sqrt{18} \\
         &= 3\sqrt{2} \approx 4.2426
  \end{align*}
  \vspace{0.2cm}
  \end{minipage}}
  \end{center}
  
  (Para comparar: $\|\vec v_1\| = \sqrt{4^2 + 1^2} = \sqrt{17} \approx 4.1231$. La velocidad sube un poco.)
  
    \item Calcule algebraicamente el ángulo entre la dirección original del movimiento ($\vec v_1$) y la dirección afectada por el viento ($\vec v_r$).
    \begin{center}
    \colorbox{red!5}{\begin{minipage}{0.9\textwidth}
    \vspace{0.2cm}
    \textbf{Desarrollo:}
    \vspace{0.2cm}
    \begin{align*}
    \cos \theta &= \frac{\vec v_1 \cdot \vec v_r}{\|\vec v_1\| \|\vec v_r\|} \\
    \vec v_1 \cdot \vec v_r &= (4)(3) + (1)(3) + (0)(0) = 15 \\
    \|\vec v_1\| &= \sqrt{17}, \quad \|\vec v_r\| = 3\sqrt{2} \\
    \cos \theta &= \frac{15}{\sqrt{17} \cdot 3\sqrt{2}} = \frac{5}{\sqrt{34}} \\
    \theta &= \arccos\left(\frac{5}{\sqrt{34}}\right) \\
    \intertext{Aproximando:}
    \sqrt{34} &\approx 5.83095, \quad \cos\theta \approx 0.85749 \\
    \theta &= \cos^{-1}(0.85749) \approx 31.0^\circ
    \end{align*}
    \vspace{0.2cm}
    \end{minipage}}
    \end{center}
    
  \item Represente en Geogebra los 3 vectores y el ángulo determinado en el paso 3.
  \begin{center}
  \colorbox{gray!5}{\parbox{0.95\textwidth}{\vspace{0.2cm}\centering
  \textbf{Enlace:} \url{https://www.geogebra.org/calculator/eeqvmjxt}\vspace{0.2cm}}}
  \end{center}
  
  \item Explique con sus palabras cómo el viento afecta la dirección y la magnitud del movimiento.
  
  \begin{center}
  \colorbox{yellow!5}{\begin{minipage}{0.9\textwidth}
  \vspace{0.3cm}
  El viento modifica tanto la dirección como la magnitud del movimiento. La velocidad aumenta ligeramente de 4.12 a 4.24 unidades, y la dirección se desvía aproximadamente 31° respecto al movimiento original del personaje.
  \vspace{0.3cm}
  \end{minipage}}
  \end{center}
\end{enumerate}

\end{document}
