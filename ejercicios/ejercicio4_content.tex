% Contenido del Ejercicio 4 sin preámbulo

\begin{center}
\colorbox{gray!10}{\parbox{0.95\textwidth}{
\vspace{0.3cm}
\centering
\textbf{\large Enunciado}
\vspace{0.2cm}

\begin{minipage}{0.9\textwidth}
En un videojuego 3D, un personaje se mueve en la dirección del vector
\[ \vec v_1 = (4,\,1,\,0) \]
y el viento aplica una fuerza representada por el vector
\[ \vec v_2 = (-1,\,2,\,0). \]
\end{minipage}
\vspace{0.3cm}
}}
\end{center}

\vspace{0.5em}
\noindent Realice lo siguiente:

\begin{enumerate}[label=\textbf{\arabic*.}, itemsep=0.8em, leftmargin=2em]
  \item Calcule el vector resultante del movimiento real del personaje $\big(\vec v_r = \vec v_1 + \vec v_2\big)$.
  \begin{center}
  \colorbox{blue!5}{\begin{minipage}{0.9\textwidth}
  \vspace{0.2cm}
  \textbf{Desarrollo:}
  \vspace{0.2cm}
  \begin{align*}
  \vec v_r &= \vec v_1 + \vec v_2 \\
       &= (4,1,0) + (-1,2,0) \\
       &= (3,3,0)
  \end{align*}
  \vspace{0.2cm}
  \end{minipage}}
  \end{center}
  
  \item Determine la velocidad total (norma de $\vec v_r$).
  \begin{center}
  \colorbox{green!5}{\begin{minipage}{0.9\textwidth}
  \vspace{0.2cm}
  \textbf{Desarrollo:}
  \vspace{0.2cm}
  \begin{align*}
  \|\vec v_r\| &= \sqrt{3^2 + 3^2 + 0^2} \\
         &= \sqrt{18} \\
         &= 3\sqrt{2} \approx 4.2426
  \end{align*}
  \vspace{0.2cm}
  \end{minipage}}
  \end{center}
  
  (Para comparar: $\|\vec v_1\| = \sqrt{4^2 + 1^2} = \sqrt{17} \approx 4.1231$. La velocidad sube un poco.)
  
    \item Calcule algebraicamente el ángulo entre la dirección original del movimiento ($\vec v_1$) y la dirección afectada por el viento ($\vec v_r$).
    \begin{center}
    \colorbox{red!5}{\begin{minipage}{0.9\textwidth}
    \vspace{0.2cm}
    \textbf{Desarrollo:}
    \vspace{0.2cm}
    \begin{align*}
    \cos \theta &= \frac{\vec v_1 \cdot \vec v_r}{\|\vec v_1\| \|\vec v_r\|} \\
    \vec v_1 \cdot \vec v_r &= (4)(3) + (1)(3) + (0)(0) = 15 \\
    \|\vec v_1\| &= \sqrt{17}, \quad \|\vec v_r\| = 3\sqrt{2} \\
    \cos \theta &= \frac{15}{\sqrt{17} \cdot 3\sqrt{2}} = \frac{5}{\sqrt{34}} \\
    \theta &= \arccos\left(\frac{5}{\sqrt{34}}\right) \\
    \intertext{Aproximando:}
    \sqrt{34} &\approx 5.83095, \quad \cos\theta \approx 0.85749 \\
    \theta &= \cos^{-1}(0.85749) \approx 31.0^\circ
    \end{align*}
    \vspace{0.2cm}
    \end{minipage}}
    \end{center}
    
  \item Represente en Geogebra los 3 vectores y el ángulo determinado en el paso 3.
  \begin{center}
  \colorbox{gray!5}{\parbox{0.95\textwidth}{\vspace{0.2cm}\centering
  \textbf{Enlace:} \url{https://www.geogebra.org/calculator/eeqvmjxt}\vspace{0.2cm}}}
  \end{center}
  
  \item Explique con sus palabras cómo el viento afecta la dirección y la magnitud del movimiento.
  
  \begin{center}
  \colorbox{yellow!5}{\begin{minipage}{0.9\textwidth}
  \vspace{0.3cm}
  El viento modifica tanto la dirección como la magnitud del movimiento. La velocidad aumenta ligeramente de 4.12 a 4.24 unidades, y la dirección se desvía aproximadamente 31° respecto al movimiento original del personaje.
  \vspace{0.3cm}
  \end{minipage}}
  \end{center}
\end{enumerate}