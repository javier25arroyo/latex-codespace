\documentclass{article}
\usepackage[utf8]{inputenc}
\usepackage[spanish]{babel}
\usepackage{amsmath}
\usepackage{amssymb}
\usepackage{xcolor}
\usepackage{graphicx}
\usepackage{enumitem}
\usepackage{geometry}
\usepackage{fancyhdr}
\usepackage{hyperref}

\geometry{a4paper, margin=2.5cm}
\definecolor{vectorcolor}{RGB}{31, 119, 180}
\definecolor{resultcolor}{RGB}{44, 160, 44}
\definecolor{formulacolor}{RGB}{214, 39, 40}

\title{\Large \textbf{Ejercicio 1: Análisis Vectorial de un Triángulo}}
\author{Javier A.}
\date{}

\pagestyle{fancy}
\fancyhf{}
\rhead{\textit{Ejercicio 1}}
\lhead{\textit{Análisis Vectorial}}
\cfoot{\thepage}

\begin{document}

\maketitle

\begin{center}
\colorbox{gray!10}{\parbox{0.95\textwidth}{
\vspace{0.3cm}
\centering
\textbf{\large Ejercicio 1}
\vspace{0.2cm}

\begin{minipage}{0.9\textwidth}
Un triángulo está definido por los puntos $A(1,2,0)$, $B(4,6,0)$ y $C(3,2,5)$. A partir de esta información, realice lo siguiente:
\end{minipage}
\vspace{0.3cm}
}}
\end{center}

\begin{enumerate}[label=\textbf{\arabic*.}, itemsep=1em, leftmargin=2em]
    \item \textcolor{vectorcolor}{\textbf{Represente los vectores $\overrightarrow{AB}$, $\overrightarrow{AC}$, y $\overrightarrow{BC}$.}}

    \begin{center}
    \colorbox{blue!5}{\begin{minipage}{0.9\textwidth}
    \vspace{0.3cm}
    \textbf{Desarrollo para verificar los resultados:}
    \begin{gather*}
        \overrightarrow{AB} = (4-1,  6-2, 0-0) \\
        \overrightarrow{AB} = \boxed{(3, 4, 0)} \\[0.5em]
        \overrightarrow{AC} = (3-1,  2-2,  5-0) \\
        \overrightarrow{AC} = \boxed{(2, 0, 5)} \\[0.5em]
        \overrightarrow{BC} = (3-4,2-6,5-0) \\
        \overrightarrow{BC} = \boxed{(-1, -4, 5)}
    \end{gather*}
    \vspace{0.2cm}
    \end{minipage}}
    \end{center}
    \item \textcolor{vectorcolor}{\textbf{Calcule la longitud (norma) de cada vector.}}
    
    \begin{center}
    \colorbox{green!5}{\begin{minipage}{0.9\textwidth}
    \vspace{0.3cm}
    \textbf{Desarrollo de las normas:}
    \begin{align*}
        \|\vec{AB}\| &= \sqrt{3^2 + 4^2 + 0^2} \\
        &= \sqrt{9 + 16 + 0} \\
        &= \sqrt{25} \\
        &= \boxed{5} \\[0.8em]
        \|\vec{AC}\| &= \sqrt{2^2 + 0^2 + 5^2} \\
        &= \sqrt{4 + 0 + 25} \\
        &= \sqrt{29} \\
        &\approx \boxed{5.385} \\[0.8em]
        \|\vec{BC}\| &= \sqrt{(-1)^2 + (-4)^2 + 5^2} \\
        &= \sqrt{1 + 16 + 25} \\
        &= \sqrt{42} \\
        &\approx \boxed{6.480}
    \end{align*}
    \vspace{0.2cm}
    \end{minipage}}
    \end{center}
    \item \textcolor{vectorcolor}{\textbf{Determine algebraicamente el ángulo entre los vectores $\overrightarrow{AB}$ y $\overrightarrow{AC}$.}}
    
    \begin{center}
    \colorbox{red!5}{\begin{minipage}{0.9\textwidth}
    \vspace{0.3cm}
    \textbf{Desarrollo del producto punto:}
    
    \fcolorbox{red!15}{white}{\begin{minipage}{0.95\textwidth}
    \vspace{0.2cm}
    \begin{center}
    \textcolor{formulacolor}{$\vec{u}\cdot\vec{v} = x_u x_v + y_u y_v + z_u z_v$}
    \end{center}
    \vspace{0.2cm}
    \end{minipage}}
    
    \vspace{0.4cm}
    Sustituyendo con $\vec{AB}=(3,4,0)$ y $\vec{AC}=(2,0,5)$:
    \begin{align*}
        \vec{AB}\cdot\vec{AC} &= (3)(2) + (4)(0) + (0)(5) \\
        &= 6 + 0 + 0 \\
        &= \boxed{6}
    \end{align*}
    \vspace{0.2cm}
    
    \begin{center}
    \fcolorbox{gray!20}{white}{\begin{minipage}{0.8\textwidth}
    \vspace{0.2cm}
    \centering
    \textbf{Resultado:} $\vec{AB}\cdot\vec{AC} = 6$
    \vspace{0.2cm}
    \end{minipage}}
    \end{center}
    \vspace{0.2cm}
    \end{minipage}}
    \end{center}

    \item \textcolor{vectorcolor}{\textbf{Interprete el resultado: ¿qué tipo de triángulo es (agudo, recto u obtuso)?}}
    
    \begin{center}
    \colorbox{yellow!5}{\begin{minipage}{0.9\textwidth}
    \vspace{0.3cm}
    Para interpretar el tipo de triángulo, analizamos el ángulo $\theta$ entre $\vec{AB}$ y $\vec{AC}$ usando la relación entre producto punto y ángulo:
    
    \fcolorbox{orange!15}{white}{\begin{minipage}{0.95\textwidth}
    \vspace{0.2cm}
    \begin{center}
    \textcolor{formulacolor}{$\vec{AB}\cdot\vec{AC} = \|\vec{AB}\|\,\|\vec{AC}\|\cos(\theta)$}
    \end{center}
    \vspace{0.2cm}
    \end{minipage}}
    
    \vspace{0.4cm}
    Con $\vec{AB}\cdot\vec{AC}=6$, $\|\vec{AB}\|=5$ y $\|\vec{AC}\|=\sqrt{29}$, se tiene:
    \begin{align*}
        6 &= (5)(\sqrt{29})\cos(\theta) \\
        \cos(\theta) &= \frac{6}{5\sqrt{29}} \approx 0.223 \\
        \theta &= \cos^{-1}(0.223) \approx \boxed{77.1^\circ}
    \end{align*}
    
    \vspace{0.4cm}
    \begin{center}
    \fcolorbox{blue!10}{white}{\begin{minipage}{0.95\textwidth}
    \vspace{0.2cm}
    \centering
    \textbf{Criterio de clasificación:}
    \vspace{0.2cm}
    
    \begin{tabular}{lcl}
        Si $\theta < 90^\circ$ & $\rightarrow$ & triángulo agudo \\
        Si $\theta = 90^\circ$ & $\rightarrow$ & triángulo recto \\
        Si $\theta > 90^\circ$ & $\rightarrow$ & triángulo obtuso
    \end{tabular}
    \vspace{0.2cm}
    \end{minipage}}
    \end{center}
    
    \vspace{0.4cm}
    \begin{center}
    \fcolorbox{resultcolor!20}{white}{\begin{minipage}{0.8\textwidth}
    \vspace{0.2cm}
    \centering
    \textbf{Conclusión:} Como $77.1^\circ < 90^\circ$, el triángulo es \textbf{AGUDO}.
    \vspace{0.2cm}
    \end{minipage}}
    \end{center}
    \vspace{0.3cm}
    \end{minipage}}
    \end{center}
    \item \textcolor{vectorcolor}{\textbf{Represente el triángulo en Geogebra utilizando la herramienta de vectores.}}
    
    \begin{center}
    \colorbox{gray!5}{\begin{minipage}{0.9\textwidth}
    \vspace{0.3cm}
    \begin{center}
    \fbox{\parbox{0.95\textwidth}{
    \vspace{0.3cm}
    \begin{center}
    \textit{Representar el triángulo en Geogebra con los puntos:}\\
    \vspace{0.2cm}
    $A(1,2,0)$, $B(4,6,0)$ y $C(3,2,5)$\\
    \vspace{0.3cm}
    Dibujar los vectores $\vec{AB}$, $\vec{AC}$ y $\vec{BC}$ y visualizar el ángulo de $77.1^\circ$\\
    entre $\vec{AB}$ y $\vec{AC}$.\\
    \vspace{0.3cm}
        	extbf{Nota:} Adjuntar el enlace o archivo de Geogebra correspondiente.\\
        \vspace{0.2cm}
        	extbf{Enlace:} \url{https://www.geogebra.org/3d/tmjswe7b}
    \end{center}
    \vspace{0.3cm}
    }}
    \end{center}
    \vspace{0.3cm}
    \end{minipage}}
    \end{center}
\end{enumerate}

\end{document}
