\documentclass{article}
\usepackage[utf8]{inputenc}
\usepackage[spanish]{babel}
\usepackage{amsmath}
\usepackage{amssymb}

\title{Ejercicio 1}
\author{Tu Nombre}
\date{\today}

\begin{document}

\maketitle

\section*{Ejercicio 1}
Un triángulo está definido por los puntos $A(1,2,0)$, $B(4,6,0)$ y $C(3,2,5)$. A partir de esta información, realice lo siguiente:

\begin{enumerate}
    \item Represente los vectores $\overrightarrow{AB}$, $\overrightarrow{AC}$, y $\overrightarrow{BC}$.

    Desarrollo para verificar los resultados:
    \begin{gather*}
        \overrightarrow{AB}= (4-1,  6-2, 0-0) \\
        \overrightarrow{AB}=(3, 4, 0) \\
        \overrightarrow{AC}= (3-1,  2-2,  5-0) \\
        \overrightarrow{AC}=(2, 0,5) \\
        \overrightarrow{BC}= (3-4,2-6,5-0) \\
        \overrightarrow{BC}=(-1, -4,5)
    \end{gather*}
    \item Calcule la longitud (norma) de cada vector.

    Desarrollo de las normas:
    \begin{gather*}
        \|\vec{AB}\| = \sqrt{3^2 + 4^2 + 0^2} \\
        = \sqrt{9 + 16 + 0} \\
        = \sqrt{25} \\
        = 5 \\
        \vec{AC}: \\
        \|\vec{AC}\| = \sqrt{2^2 + 0^2 + 5^2} = \sqrt{4 + 0 + 25} = \sqrt{29} \\
        \sqrt{29} \approx 5.385 \\
        \vec{BC}: \\
        \|\vec{BC}\| = \sqrt{(-1)^2 + (-4)^2 + 5^2} = \sqrt{1 + 16 + 25} = \sqrt{42} \\
        \sqrt{42} \approx 6.480
    \end{gather*}
    \item Determine algebraicamente el ángulo entre los vectores $\overrightarrow{AB}$ y $\overrightarrow{AC}$.\\
    Desarrollo del producto punto:
    \begin{gather*}
        \vec{u}\cdot\vec{v} = x_u x_v + y_u y_v + z_u z_v
    \end{gather*}
    Sustituyendo con $\vec{AB}=(3,4,0)$ y $\vec{AC}=(2,0,5)$:
    \begin{gather*}
        \vec{AB}\cdot\vec{AC} = (3)(2) + (4)(0) + (0)(5) \\
        = 6 + 0 + 0 \\
        = 6
    \end{gather*}
    \noindent\textbf{Resultado:} $\vec{AB}\cdot\vec{AC} = 6$.

    \item Interprete el resultado: ¿qué tipo de triángulo es (agudo, recto u obtuso)?\\
    Para interpretar el tipo de triángulo, analizamos el ángulo $\theta$ entre $\vec{AB}$ y $\vec{AC}$ usando la relación entre producto punto y ángulo:
    \begin{gather*}
        \vec{AB}\cdot\vec{AC} = \|\vec{AB}\|\,\|\vec{AC}\|\cos(\theta)
    \end{gather*}
    Con $\vec{AB}\cdot\vec{AC}=6$, $\|\vec{AB}\|=5$ y $\|\vec{AC}\|=\sqrt{29}$, se tiene:
    \begin{gather*}
        6 = (5)(\sqrt{29})\cos(\theta) \quad \Rightarrow \quad \cos(\theta)=\frac{6}{5\sqrt{29}} \approx 0.223 \\
        	heta = \cos^{-1}(0.223) \approx 77.1^\circ
    \end{gather*}
    Criterio:
    \begin{itemize}
        \item $\theta<90^\circ$: triángulo agudo
        \item $\theta=90^\circ$: triángulo recto
        \item $\theta>90^\circ$: triángulo obtuso
    \end{itemize}
    Como $77.1^\circ<90^\circ$, el triángulo es \textbf{agudo}.
    \item Represente el triángulo en Geogebra utilizando la herramienta de vectores y represente el ángulo calculado en el punto 3. Se debe adjuntar el enlace o archivo del Geogebra correspondiente.
\end{enumerate}

\end{document}
