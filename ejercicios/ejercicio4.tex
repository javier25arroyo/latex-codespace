\documentclass{article}
\usepackage[utf8]{inputenc}
\usepackage[spanish]{babel}
\usepackage{amsmath, amssymb}
\usepackage{xcolor}
\usepackage{graphicx}
\usepackage{enumitem}
\usepackage{geometry}
\usepackage{fancyhdr}
\usepackage{hyperref}
\usepackage{bookmark} % añadido para evitar la advertencia de rerunfilecheck
\hypersetup{colorlinks=true, urlcolor=blue, linkcolor=black}

\geometry{a4paper, margin=2.5cm}
% Colores de apoyo
\definecolor{vectorcolor}{RGB}{31,119,180}
\definecolor{resultcolor}{RGB}{44,160,44}
\definecolor{formulacolor}{RGB}{214,39,40}
\title{\Large \textbf{Ejercicio 4: Vectores en un videojuego 3D}}
\author{Javier A.}
\date{}

\pagestyle{fancy}
\fancyhf{}
\rhead{\textit{Ejercicio 4}}
\lhead{\textit{Vectores y Movimiento}}
\cfoot{\thepage}

\begin{document}

\maketitle

\begin{center}
\colorbox{gray!10}{\parbox{0.95\textwidth}{
\vspace{0.3cm}
\centering
\textbf{\large Enunciado}
\vspace{0.2cm}

\begin{minipage}{0.9\textwidth}
En un videojuego 3D, un personaje se mueve en la dirección del vector
\[ \vec v_1 = (4,\,1,\,0) \]
y el viento aplica una fuerza representada por el vector
\[ \vec v_2 = (-1,\,2,\,0). \]
\end{minipage}
\vspace{0.3cm}
}}
\end{center}

\vspace{0.5em}
\noindent Realice lo siguiente:

\begin{enumerate}[label=\textbf{\arabic*.}, itemsep=0.8em, leftmargin=2em]
  \item Calcule el vector resultante del movimiento real del personaje $\big(\vec v_r = \vec v_1 + \vec v_2\big)$.
  \begin{center}
  \colorbox{blue!5}{\begin{minipage}{0.9\textwidth}
    	\vspace{0.2cm}
    		\textbf{Desarrollo:}
    	\vspace{0.2cm}
  \begin{align*}
  \vec v_r &= \vec v_1 + \vec v_2 \\
       &= (4,1,0) + (-1,2,0) \\
       &= (4+(-1), 1+2, 0+0) \\
       &= (3,3,0)
  \end{align*}
  \end{minipage}}
  \end{center}
  \item Determine la velocidad total (norma de $\vec v_r$).
  \begin{center}
  \colorbox{green!5}{\begin{minipage}{0.9\textwidth}
    	\vspace{0.2cm}
    		\textbf{Desarrollo:}
    	\vspace{0.2cm}
  \begin{align*}
  \|\vec v_r\| &= \sqrt{3^2 + 3^2 + 0^2} \\
         &= \sqrt{9 + 9 + 0} \\
         &= \sqrt{18} \\
         &= 3\sqrt{2} \approx 4.2426
  \end{align*}
  \end{minipage}}
  \end{center}
    \item Calcule algebraicamente el ángulo entre la dirección original del movimiento ($\vec v_1$) y la dirección afectada por el viento ($\vec v_r$).
    \begin{center}
    \colorbox{red!5}{\begin{minipage}{0.9\textwidth}
    \vspace{0.2cm}
    	\textbf{Desarrollo:}
    \vspace{0.2cm}
    \begin{align*}
    \cos \theta &= \frac{\vec v_1 \cdot \vec v_r}{\|\vec v_1\| \|\vec v_r\|} \\
    \vec v_1 \cdot \vec v_r &= (4)(3) + (1)(3) + (0)(0) = 12 + 3 + 0 = 15 \\
    \|\vec v_1\| &= \sqrt{4^2 + 1^2 + 0^2} = \sqrt{16 + 1 + 0} = \sqrt{17} \\
    \|\vec v_r\| &= 3\sqrt{2} \\
    \cos \theta &= \frac{15}{\sqrt{17} \cdot 3\sqrt{2}} = \frac{15}{3\sqrt{34}} = \frac{5}{\sqrt{34}} \\
    	\theta &= \arccos\left(\frac{5}{\sqrt{34}}\right) \\
    \intertext{Aproximando:}
    \sqrt{34} &\approx 5.83095 \\
    \cos\theta &\approx \frac{5}{5.83095} \approx 0.85749 \\
      	\theta &= \cos^{-1}(0.85749) \approx 31.0^\circ
    \end{align*}
    \end{minipage}}
    \end{center}
    \begin{center}
    \fcolorbox{formulacolor!20}{white}{\parbox{0.9\textwidth}{\centering Resultado: $\theta = \arccos\big(\frac{5}{\sqrt{34}}\big) \approx 31.0^\circ$}}
    \end{center}
  \item Represente en Geogebra los 3 vectores y el ángulo determinado en el paso 3. Se debe adjuntar el enlace o archivo de Geogebra correspondiente.
  \begin{center}
  \colorbox{gray!5}{\parbox{0.95\textwidth}{\vspace{0.2cm}\centering
  	\textbf{Enlace:} \url{https://www.geogebra.org/calculator/eeqvmjxt}\vspace{0.2cm}}}
  \end{center}
	\item Explique con sus palabras cómo el viento afecta la dirección y la magnitud del movimiento.
\end{enumerate}

\end{document}

